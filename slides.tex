%%This is a very basic article template.
%%There is just one section and two subsections.
\documentclass{beamer}

\input{preamble}

\tikzstyle{simplenode}=[shape=circle,draw=black,
top color=ubcgray!5,bottom color=ubcgray!40,
semithick]
\tikzstyle{basicnode}=[simplenode,circular drop shadow]

\title[Biomedical Image Segmentation by Deep Learning]{White Matter Lesion
Segmentation\\ by Deep Learning}

\author{Tom Brosch}

\institute[Universities of British Columbia]
{
MS/MRI Research Group\\
Electrical and Computer Engineering\\
University of British Columbia
}

%\date{Ph.D. Proposal Defense}
\date{February 22, 2016}

\begin{document}

\begin{frame}
\titlepage
\end{frame}

\makeatletter
\setbeamertemplate{enumerate items}{
    \llap{\begin{pgfpicture}{-1ex}{-0.8ex}{1ex}{1ex}
      {\pgftransformscale{0.9}\pgftext{\beamer@usesphere{section number
      projected}{tocsphere}}} \pgftext{%
        \usebeamerfont*{section number projected}%
        \usebeamercolor{section number projected}%
        \color{fg!90!bg}%
        \insertenumlabel}
    \end{pgfpicture}%
    %\kern0.15ex
    }%
}
\makeatother

\begin{frame}{Outline}
\tableofcontents
\end{frame}

\section{Introduction}

\subsection{Motivation}

\begin{frame}{Motivation}
\begin{itemize}
\item White matter lesions are a hallmark of MS
\item Imaging biomarkers for tracking disease progression and treatment effect,
e.g.:
\begin{itemize}
\item Lesion load
\item Lesion count
\end{itemize}
\item Lesions vary greatly in size, shape, intensity and location
\item<2->[$\Rightarrow$] \alert{Automatic and accurate segmentation
challenging!}
\end{itemize}
\end{frame}

\subsection{Related Work}

\begin{frame}{Related Work: Unsupervised methods}
\begin{description}
\item[Approach 1] 
Divide the intensities of brain MRIs into the four clusters
CSF, grey matter, white matter, and lesion voxels (e.g., lesion-TOADS)
\item[Approach 2] Model the intensity distribution of CSF, gray matter, and
white matter; lesions are outlines of that distribution (e.g., EMS)
\end{description}
\begin{itemize}
\pros<2-> Don't require labelled data for training
\pros<2-> Segmentation based on a per-subject per-image model
\pros<2-> Robust to inter-subject and inter-image variability
\cons<3-> Not specific to lesions
\cons<3-> Sensitive to non-lesion outliers such as blood vessels and partial
volume effects
\end{itemize}
\end{frame}

\begin{frame}{Related Work: Supervised methods}
\begin{description}
\item[Approach] Segmentation as a voxel classification problem
\item[Pipeline] Extract features $\Rightarrow$ classify voxel $\Rightarrow$
post-processing
\item[Features] Pre-defined or learned from unlabelled images
\item[Classifier] Random-forest, SVM, neural networks, library-based
\item[Post-processing] Markov random field, removal of isolated voxels
\end{description}
\begin{itemize}
\pros<2-> Lesion-specific
\cons<3-> Sensitive to inter-subject and inter-scanner variability
\cons<3-> Require large amounts of training data
\end{itemize}
\end{frame}

\begin{frame}{Related Work: End-to-end learning}
\begin{description}
\item[Main idea] Joint learning of features and classifier from labelled data
using a convolutional neural network (CNN)
\item[Approach 1] Classify image patches using a CNN (patch-based)
\item[Approach 2] Feed in entire volumes into a CNN\\ (fully convolutional)
\end{description}
\begin{itemize}
\pros<2-> Lesion-specific
\pros<2-> Learning of features that are tuned to the data and segmentation
task
\pros<2-> Can learn features that are robust to inter-subject and inter-scanner
variability
\cons<3-> Requires large amounts of training data
\cons<3-> Training and inference of patch-based approaches can be prohibitively
slow
\end{itemize}
\end{frame}

\section{Method}

\subsection{Overview}

\begin{frame}{Method overview}
\begin{itemize}
\item Training based method
\item Requires pairs of MRIs and corresponding segmentations
\item Learns parameters to perform segmentation
\item Replicate slides from my talk
\item Simple neuron slide with knobs
\item Patch-based neural network for training
\item Fully convolutional
\end{itemize}
\end{frame}

\subsection{Training}

\begin{frame}{Training Framework}
\begin{itemize}
\item Trained by gradient descent
\item Gradient calculated using back-propagation
\item What is a good objective function?
\end{itemize}
\end{frame}

\begin{frame}{Objective Function}
\begin{itemize}
\item Sum of squared differences
\begin{itemize}
\item Simple and often used
\item Puts equal emphasis on lesion and non-lesion voxels
\item However, lesion voxels make up only few percent of all voxels
\item<2->[$\Rightarrow$] \alert{Can learn to ignore lesion voxels!} 
\end{itemize}
\item<3-> Weighted combination of sensitivity and specificity
\begin{itemize}
\item Can be used together to assess segmentation performance
\item Allows weighting of the importance of both terms
\item Formalized to yield smooth gradients allows faster convergence
\item<4->[$\Rightarrow$] \alert{New objective function facilitates the use of
fully convolutional neural networks for lesion segmentation!}
\end{itemize}
\end{itemize}
\end{frame}

\subsection{Network architecture}

\begin{frame}{Network Architecture /1}
\begin{itemize}
\item 3-layer network (use same style as journal paper)
\item Show result
\end{itemize}
\end{frame}

\begin{frame}{Network Architecture /2}
\begin{itemize}
\item 7-layer CEN without shortcuts
\item Show result
\end{itemize}
\end{frame}

\begin{frame}{Network Architecture /3}
\begin{itemize}
\item 7-layer CEN with shortcuts
\item Show result
\end{itemize}
\end{frame}

\section{Evaluation}

\subsection{Overview}

\begin{frame}{Overview}
\begin{itemize}
\item Evaluated on two data sets
\end{itemize}

\end{frame}

\begin{frame}{Evaluation Measures}
\begin{itemize}
\item DSC
\item Lesion true positives rate
\item Lesion false positive rate
\item Volume difference
\end{itemize}
\end{frame}

\subsection{MICCAI 2008 Lesion Segmentation Challenge}

\begin{frame}{MICCAI 2008 Lesion Segmentation Challenge /1}
\begin{itemize}
\item Explain the data
\end{itemize}
\end{frame}

\begin{frame}{MICCAI 2008 Lesion Segmentation Challenge /1}
Results table
\end{frame}

\subsection{Bravo data set}

\begin{frame}{Bravo data set /1}
\begin{itemize}
\item Explain data set
\item Explain stratification
\end{itemize}
\end{frame}

\begin{frame}{Bravo data set /2}
Results table
\end{frame}

\begin{frame}{Bravo data set /2}
Plots one by one.
\end{frame}

\section{Conclusions}

\begin{frame}{Conclusions}
\begin{itemize}
\item Look at conclusions
\end{itemize}
\end{frame}

\begin{frame}
Thank you. Questions?
\end{frame}

\end{document}
